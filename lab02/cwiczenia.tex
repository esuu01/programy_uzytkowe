\documentclass[11pt, a4paper, titlepage]{article}
\usepackage[left=3.5cm, right=2.5cm, top=2.5cm, bottom=2.5cm]{geometry}
\usepackage[MeX]{polski}
\usepackage[utf8]{inputenc}
\usepackage{graphicx}
\usepackage{enumerate}
\usepackage{amsmath} %pakiet matematyczny
\usepackage{amssymb} %pakiet dodatkowych symboli

\title{Przepis na szarlotkę}
\author{Maciej Mierzejewski}
\date{15 Październik 2022}

\begin{document}
\maketitle

\section{Wstęp}
\paragraph{Pyszna i prosta do zrobienia szarlotka na kruchym cieście. Każdemu się udaje i wszystkim smakuje! Dla ekstra kruchości część masła można zastąpić smalcem.}

\section{Składniki \underline{(na 12 porcji)}}
\subsection{Jabłka}
\begin{itemize}
\item 1,5 kg jabłek (na szarlotkę najlepiej twardych i kwaśnych, np. szara reneta)
\item 5 łyżek cukru
\item 1/2 łyżeczki cynamonu
\end{itemize}

\subsection{Ciasto}
\begin{itemize}
\item 300g mąki
\item 250 g zimnego masła (50 g masła można zastąpić smalcem)
\item 1,5 łyżeczki proszku do pieczenia
\item 5 łyżek cukru
\item 1 łyżka cukru wanilinowego
\item 1 jajko
\item Do posypania: cukier puder
\end{itemize}

\section{Przygotowanie}
\subsection{Jabłka}
\begin{enumerate}
\item Jabłka obrać, pokroić na ćwiartki i wyciąć gniazda nasienne. Pokroić na mniejsze kawałki i włożyć do szerokiego garnka lub na głęboką patelnię.
\item Dodać cukier i cynamon i smażyć przez ok. 20 minut co chwilę mieszając, aż jabłka zmiękną i zaczną się rozpadać.
\end{enumerate}

\subsection{Ciasto}
\begin{enumerate}
\item Do mąki dodać pokrojone w kostkę \textbf{zimne} masło, proszek do pieczenia, cukier i cukier wanilinowy.
\item Składniki połączyć w jednolite ciasto \textit{(mikserem lub ręcznie)}, pod koniec dodać jajko \textit{(ciasto będzie dość miękkie)}.
\item Podzielić je na pół i włożyć obie połówki do zamrażarki na ok. 15 minut.
\end{enumerate}
\subsection{Pieczenie}
\begin{enumerate}
\item Piekarnik nagrzać do \textbf{180 st C}. Przygotować niedużą formę *.
\item Wyjąć jedną połówkę ciasta z zamrażarki, pokroić nożem na plasterki i wylepić nimi spód formy. Następnie wyłożyć na to jabłka.
\item Pozostałe ciasto zetrzeć na tarce bezpośrednio na jabłka \textit{(lub pokroić ciasto na plasterki i ułożyć na wierzchu)}.
\item Wstawić do piekarnika i piec przez \textbf{ok. 50 minut} lub na złoty kolor. Upieczoną szarlotkę przestudzić i posypać cukrem pudrem.
\end{enumerate}

\subsubsection{Wskazówki}
\begin{enumerate}[I.]
\item {\fontsize{9pt}{\baselineskip}\selectfont np. prostokątną formę 21 x 27 cm lub kwadratową 24 x 24 cm lub tortownicę o średnicy 26 cm}\par
\end{enumerate}

\end{document}
